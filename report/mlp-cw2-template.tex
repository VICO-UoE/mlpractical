%% Template for MLP Coursework 2 / 13 November 2023

%% Based on  LaTeX template for ICML 2017 - example_paper.tex at 
%%  https://2017.icml.cc/Conferences/2017/StyleAuthorInstructions

\documentclass{article}
\usepackage[T1]{fontenc}
\usepackage{amssymb,amsmath}
\usepackage{txfonts}
\usepackage{microtype}

% For figures
\usepackage{graphicx}
\usepackage{subcaption} 

% For citations
\usepackage{natbib}

% For algorithms
\usepackage{algorithm}
\usepackage{algorithmic}

% the hyperref package is used to produce hyperlinks in the
% resulting PDF.  If this breaks your system, please commend out the
% following usepackage line and replace \usepackage{mlp2017} with
% \usepackage[nohyperref]{mlp2017} below.
\usepackage{hyperref}
\usepackage{url}
\urlstyle{same}

\usepackage{color}
\usepackage{booktabs} % To thicken table lines
\usepackage{multirow} % Multirow cells in table

% Packages hyperref and algorithmic misbehave sometimes.  We can fix
% this with the following command.
\newcommand{\theHalgorithm}{\arabic{algorithm}}


% Set up MLP coursework style (based on ICML style)
\usepackage{mlp2022}
\mlptitlerunning{MLP Coursework 2 (\studentNumber)}
\bibliographystyle{icml2017}
\usepackage{bm,bbm}
\usepackage{soul}

\DeclareMathOperator{\softmax}{softmax}
\DeclareMathOperator{\sigmoid}{sigmoid}
\DeclareMathOperator{\sgn}{sgn}
\DeclareMathOperator{\relu}{relu}
\DeclareMathOperator{\lrelu}{lrelu}
\DeclareMathOperator{\elu}{elu}
\DeclareMathOperator{\selu}{selu}
\DeclareMathOperator{\maxout}{maxout}
\newcommand{\bx}{\bm{x}}




\definecolor{red}{rgb}{0.95,0.4,0.4}
\definecolor{blue}{rgb}{0.4,0.4,0.95}
\definecolor{orange}{rgb}{1, 0.65, 0}

\newcommand{\youranswer}[1]{{\color{red} \bf[#1]}} %your answer: 


%% START of YOUR ANSWERS
%% REPLACE sXXXXXXX with your student number
\def\studentNumber{sXXXXXXX}


%% START of YOUR ANSWERS
%% Add answers to the questions below, by replacing the text inside the brackets {} for \youranswer{ "Text to be replaced with your answer." }. 
%
% Do not delete the commands for adding figures and tables. Instead fill in the missing values with your experiment results, and replace the images with your own respective figures.
%
% You can generally delete the placeholder text, such as for example the text "Question Figure 3 - Replace the images ..." 
%
% There are 5 TEXT QUESTIONS. Replace the text inside the brackets of the command \youranswer with your answer to the question.
%
% There are also 3 "questions" to replace some placeholder FIGURES with your own, and 1 "question" asking you to fill in the missing entries in the TABLE provided. 
%
% NOTE! that questions are ordered by the order of appearance of their answers in the text, and not necessarily by the order you should tackle them. You should attempt to fill in the TABLE and FIGURES before discussing the results presented there. 
%
% NOTE! If for some reason you do not manage to produce results for some FIGURES and the TABLE, then you can get partial marks by discussing your expectations of the results in the relevant TEXT QUESTIONS. The TABLE specifically has enough information in it already for you to draw meaningful conclusions.
%
% Please refer to the coursework specification for more details.


%% - - - - - - - - - - - - TEXT QUESTIONS - - - - - - - - - - - - 

%% Question 1:
\newcommand{\questionOne} {
\youranswer{
The VGG_08 is a relatively shallow network with only three layers, compared to the much deeper VGG_38 model containing 38 layers. Training the models for 100 epochs, the shallower VGG_08 networks manages steadily reduce the training loss converging to a low error, whilst also performing well on the validation set (Figure 1). The accuracy of the VGG_08 model seems to plateau between 0.5-0.6 (Figure 1), perhaps implying that the network is too shallow for the CIFAR-100 classification task. The VGG_38 model, however, is unable to anything about the information in the dataset, which would enable it to perform better on classification. The loss curves seem to level out at a high value very early on in the training session, whilst the accuracy never improves beyond 0.0 (Figure 1), so the model indeed does not manage to learn anything at all about the data. This indicates that the deeper architecture is not able to use the information capacity of its parameters, to capture features in the data. In the healthy VGG_08 model, the gradients in the initial layers are given higher importance on average than the middle or last layers, illustrating how important it is for the initial layers to capture the main features in the data (Figure 2). The average gradients for the VGG_38 model stagnate at 0 in the first and middle layers, whilst spiking slightly higher in the last two reduction blocks and the last linear prediction output layer (Figure 3). Such results clearly show the presence of the vanishing gradients problem in the deeper VGG_38 model. By the chain rule in backpropagation, the gradients are multiplied through the network from the output to the first layer. In the VGG_38 network this process quite rapidly reduces the gradients to a very small number close to zero or zero, meaning that the weights will not be updated effectively or will not be updated at all. The consequences of this are the model's inability to learn features in the data and hence perform well in the classification task on the training and test sets (Figure 1).}
}

%% Question 2:
\newcommand{\questionTwo} {
\youranswer{Question 2 - Describe Batch Normalization (BN) by using equations. Consider both training and test time and explain how BN addresses the vanishing gradient problem. Note that you are not required to provide the derivation of gradients w.r.t. weights for BN weights.
    The aim of Batch Normalization is to ensure that the distribution of nonlinear inputs remains stable and does not saturate as the network trains. This change in distribution, which causes the internal nodes of a deep network to saturate is called the Internal Covariate Shift, as already mentioned. Fixing this distribution of the layer inputs as the training progresses is expected to improve the training speed. Through introducing the normalization of inputs to each layer, we reduce the damaging effects of the Internal Covariate Shift, by moving towards achieving this fixed distribution of inputs (reference). We normalize each scalar input feature independently, by making it have zero mean and unit variance, meaning that for a layer with an n-dimensional input $x= (x^{(1)} ... x^{(n)})$ each dimension is normalized as: \[\hat{x}^{(k)} = \frac{x^{(k)} - E[x^{(k)}]}{\sqrt{Var[x^{(k)}]}}\]. Here the expectation and variance can be computed over either the entire training dataset or a smaller batch set in stochastic gradient training, in which case each mini-batch produces the mean and variance estimates of every activation. In order to preserev the layer's representational capacity, we ensure that the transformation introduced by the normalization step can represent the identity transform. Two new learnable parameters, learned along with the original model parameters, are introduced for each activation $\gamma^{(k)}$, $\beta^{(k)}$, that in turn scale and shift the normalized input value:\[\gamma^{(k)} = \gamma^{(k)}\hat{x}^{(k)} + \beta^{(k)}\]. For a given mini-batch of input values to a layer, \[\Beta ={x_1,x_2,...,x_m}\] Then, as noted above, the normalized values are $\hat{x}_{1...m}$ and their corresponding linear transformations $y_{1...m}$. The Batch Normalizing transform then consists of the following outputs, based on the learned parameters $\gamma$ and $\beta$:\ \[\mu_{B} \leftarrow \frac{1}{m} \sum_{i=1}^{m} x_i    - the mini-batch mean\] ,  \[\sigma_{B}^2 \leftarrow \frac{1}{m} \sum_{i=1}^{m} (x_i-\mu_{B})^2\] - the mini-batch variance, \[\hat{x}_i \leftarrow \frac{(x_i-\mu_{B})}{\sqrt{\sigma_{B}^2 + \epsilon}}    - the normalized input value,\] \[y_i \leftarrow \gamma\hat{x}_i + \beta    - the Batch Normalizing transform\]. Where we can neglect $\epsilon$, assuming that every element of each mini-batch is sampled from the same distribution. Using this to normalize a network, we pass the normalized input to every layer instead of the original input. This transform is a differentiable function, which introduces normalized activations into the network, so that layers can continue learning on input distributions that suffer from less Internal Covariate Shift, hence preventing small changes to the parameters from amplifying into larger and suboptimal changes in activations in gradients. That is, it mitigates the vanishing gradients problem. At test time, we are interested in how well the model is performing on new, unseen data, so we do not want to transform it. Rather, come test time, we only normalize the inputs, but with the statistics of the entire dataset and not the mini-batch like so:\[\hat{x} = \frac{x - E[x]}{\sqrt{Var[x]} + \epsilon}\]. Here, the means and variances are now fixed. The linear transform with $\gamma$ and $\beta$ can also be applied, but those parameters also remain fixed. Using training set statistics to normalize 

The average length of an answer to this question would be around 2/3 of a column in a 2-column page}
}

%% Question 3:
\newcommand{\questionThree} {
\youranswer{Question 3 - Describe Residual Connections (RC) by using equations. Consider both training and test time and explain how RC address the vanishing gradient problem.

The average length of an answer to this question would be around 1/2 of a column in a 2-column page}
}

%% Question 4:
\newcommand{\questionFour} {
\youranswer{Question 4 - Present and discuss the experiment results (all of the results and not just the ones you had to fill in) in Table 1 and Figures 4 and 5 (you may use any of the other Figures if you think they are relevant to your analysis). You will have to determine what data are relevant to the discussion, and what information can be extracted from it. Also, discuss what further experiments you would have ran on any combination of VGG08, VGG38, BN, RC in order to
\begin{itemize}
    \item Improve performance of the model trained (explain why you expect your suggested experiments will help with this).
    \item Learn more about the behaviour of BN and RC (explain what you are trying to learn and how).
\end{itemize}

The average length for an answer to this question is approximately 1 of the columns in a 2-column page}
}

%% Question 5:
\newcommand{\questionFive} {
\youranswer{Question 5 - Briefly draw your conclusions based on the results from the previous sections (what are the take-away messages?) and conclude your report with a recommendation for future work. 

Good recommendations for future work also draw on the broader literature (the papers already referenced are good starting points). Great recommendations for future work are not just incremental (an example of an incremental suggestion would be: "we could also train with different learning rates") but instead also identify meaningful questions or, in other words, questions with answers that might be somewhat more generally applicable. 

For example, \citep{huang2017densely} end with \begin{quote}``Because of their compact internal representations and reduced feature redundancy, DenseNets may be good feature extractors for various computer vision tasks that build on convolutional features, e.g.,  [4,5].''\end{quote} 

while \cite{bengio1993problem} state in their conclusions that \begin{quote}``There remains theoretical questions to be considered,  such as whether the problem with simple gradient descent  discussed in this paper would be observed with  chaotic attractors that are not  hyperbolic.\\\end{quote}

The length of this question description is indicative of the average length of a conclusion section}
}

%% - - - - - - - - - - - - FIGURES - - - - - - - - - - - - 

%% Question Figure 3:
\newcommand{\questionFigureThree} {
\youranswer{Question Figure 3 - Replace this image with a figure depicting the average gradient across layers, for the VGG38 model.
%
\begin{figure}[t]
    \centering
    \includegraphics[width=\linewidth]{example-image-duck}
    \caption{Gradient Flow on VGG38}
    \label{fig:grad_flow_38}
\end{figure}
}
}

%% Question Figure 4:
\newcommand{\questionFigureFour} {
\youranswer{Question Figure 4 - Replace this image with a figure depicting the training curves for the model with the best performance across experiments you have available. (Also edit the caption accordingly).
%
\begin{figure}[t]
    \centering
    \includegraphics[width=\linewidth]{example-image-duck}
    \caption{Training curves for ? ? ?}
    \label{fig:grad_flow_bestModel}
\end{figure}
}
}

%% Question Figure 5:
\newcommand{\questionFigureFive} {
\youranswer{Question Figure 5 - Replace this image with a figure depicting the average gradient across layers, for the model with the best performance across experiments you have available. (Also edit the caption accordingly).
%
\begin{figure}[t]
    \centering
    \includegraphics[width=\linewidth]{example-image-duck}
    \caption{Gradient Flow on ? ? ?}
    \label{fig:grad_flow_bestModel}
\end{figure}
}
}

%% - - - - - - - - - - - - TABLES - - - - - - - - - - - - 

%% Question Table 1:
\newcommand{\questionTableOne} {
\youranswer{
Question Table 1 - Fill in Table 1 with the results from your experiments on 
\begin{enumerate}
    \item \textit{VGG38 BN (LR 1e-3)}, and 
    \item \textit{VGG38 BN + RC (LR 1e-2)}.
\end{enumerate}
%
\begin{table*}[t]
    \centering
    \begin{tabular}{lr|ccccc}
    \toprule
        Model                   & LR   & \# Params & Train loss & Train acc & Val loss & Val acc \\
    \midrule
        VGG08                   & 1e-3 & 60 K      &  1.74      & 51.59     & 1.95     & 46.84 \\
        VGG38                   & 1e-3 & 336 K     &  4.61      & 00.01     & 4.61     & 00.01 \\
        VGG38 BN                & 1e-3 & 338 K     &  1.17      & 0.655     & 1.77     & 0.529 \\
        VGG38 RC                & 1e-3 & 336 K     &  1.33      & 61.52     & 1.84     & 52.32 \\
        VGG38 BN + RC           & 1e-3 & 339 K     &  1.26      & 62.99     & 1.73     & 53.76 \\
        VGG38 BN                & 1e-2 & 339 K     &  1.70      & 52.28     & 1.99     & 46.72 \\
        VGG38 BN + RC           & 1e-2 &     ?     &     ?      &     ?     &    ?     &     ? \\
    \bottomrule
    \end{tabular}
    \caption{Experiment results (number of model parameters, Training and Validation loss and accuracy) for different combinations of VGG08, VGG38, Batch Normalisation (BN), and Residual Connections (RC), LR is learning rate.}
    \label{tab:CIFAR_results}
\end{table*} 
}
}

%% END of YOUR ANSWERS
%% END of YOUR ANSWERS



%% Do not change anything in this file. Add your answers to mlp-cw1-questions.tex



\begin{document} 

\twocolumn[
\mlptitle{MLP Coursework 2}
\centerline{\studentNumber}
\vskip 7mm
]

\begin{abstract} 
Deep neural networks have become the state-of-the-art 
in many standard computer vision problems thanks to their powerful
representations and availability of large labeled datasets.
While very deep networks allow for learning more levels of abstractions in their layers from the data, training these models successfully is a challenging task due to problematic gradient flow through the layers, known as vanishing/exploding gradient problem.
In this report, we first analyze this problem in VGG models with 8 and 38 hidden layers on the CIFAR100 image dataset, by monitoring the gradient flow during training. 
We explore known solutions to this problem including batch normalization or residual connections, and explain their theory and implementation details. 
Our experiments show that batch normalization and residual connections effectively address the aforementioned problem and hence enable a deeper model to outperform shallower ones in the same experimental setup.
\end{abstract} 

\section{Introduction}
\label{sec:intro}
Despite the remarkable progress of modern convolutional neural networks (CNNs) in image classification problems~\cite{simonyan2014very, he2016deep}, training very deep networks is a challenging procedure.
One of the major problems is the Vanishing Gradient Problem (VGP), a phenomenon where the gradients of the error function with respect to network weights shrink to zero, as they backpropagate to earlier layers, hence preventing effective weight updates. 
This phenomenon is prevalent and has been extensively studied in various deep neural networks including feedforward  networks~\cite{glorot2010understanding},  RNNs~\cite{bengio1993problem}, and CNNs~\cite{he2016deep}. 
Multiple solutions have been proposed to mitigate this problem by using weight initialization strategies~\cite{glorot2010understanding},
activation functions~\cite{glorot2010understanding}, input normalization~\cite{bishop1995neural},
batch normalization~\cite{ioffe2015batch}, and shortcut connections \cite{he2016deep, huang2017densely}.

This report focuses on diagnosing the VGP occurring in the VGG38 model\footnote{VGG stands for the Visual Geometry Group in the University of Oxford.} and addressing it by implementing two standard solutions.
In particular, we first study a ``broken'' network in terms of its gradient flow, L1 norm of gradients with respect to its weights for each layer and contrast it to ones in the healthy and shallower VGG08 to pinpoint the problem.
Next, we review two standard solutions for this problem,  batch normalization (BN)~\cite{ioffe2015batch} and residual connections (RC)~\cite{he2016deep} in detail and discuss how they can address the gradient problem.
We first incorporate batch normalization (denoted as VGG38+BN), residual connections (denoted as VGG38+RC),  and their combination (denoted as VGG38+BN+RC) to the given VGG38 architecture.
We train the resulting three configurations, and VGG08 and VGG38 models on CIFAR100 (pronounced as `see far 100' ) dataset and present the results.
The results show that though separate use of BN and RC does mitigate the vanishing/exploding gradient problem, therefore enabling effective training of the VGG38 model, the best results are obtained by combining both BN and RC.

%


\section{Identifying training problems of a deep CNN}
\label{sec:task1}

\begin{figure}[t]
    \begin{subfigure}{\linewidth}
        \centering
        \includegraphics[width=\linewidth]{figures/loss_plot.pdf}
        \caption{Cross entropy error per epoch}
        \label{fig:loss_curves}
    \end{subfigure}

    \begin{subfigure}{\linewidth}
        \centering
        \includegraphics[width=\linewidth]{figures/accuracy_plot.pdf}
        \caption{Classification accuracy per epoch}
        \label{fig:acc_curves}
    \end{subfigure}
    \caption{Training curves for VGG08 and VGG38 in terms of (a) cross-entropy error and (b) classification accuracy}
    \label{fig:curves}
\end{figure}

\begin{figure}[t]
    \centering
    \includegraphics[width=\linewidth]{figures/grad_flow_vgg08.pdf}
    \caption{Gradient flow on VGG08}
    \label{fig:grad_flow_08}
\end{figure}

\questionFigureThree

Concretely, training deep neural networks typically involves three steps: forward
pass, backward pass (or backpropagation algorithm~\cite{rumelhart1986learning}) and weight update.
The first step involves passing the input $\bx^{(0)}$ to the network and producing 
the network prediction and also the error value.
In detail, each layer takes in the output of the previous layer and applies
a non-linear transformation:
\begin{equation}
\label{eq.fprop}
\bx^{(l)} = f^{(l)}(\bx^{(l-1)}; W^{(l)})    
\end{equation} 
where $(l)$ denotes the $l$-th layer in $L$ layer deep network,
$f^{(l)}(\cdot,W^{(l)})$ is a non-linear transformation for layer $l$, and $W^{(l)}$ are the weights of layer $l$.
For instance, $f^{(l)}$ is typically a convolution operation followed by an activation function in convolutional neural networks.
The second step involves the backpropagation algorithm, where we calculate the gradient of an error function $E$ (\textit{e.g.} cross-entropy) for each layer's weight as follows:

\begin{equation}
    \label{eq.bprop}
\frac{\partial E}{\partial W^{(l)}} = \frac{\partial E}{\partial \bx^{(L)}} \frac{\partial \bx^{(L)}}{\partial \bx^{(L-1)}} \dots \frac{\partial \bx^{(l+1)}}{\partial \bx^{(l)}}\frac{\partial \bx^{(l)}}{\partial W^{(l)}}.
\end{equation}

This step includes consecutive tensor multiplications between multiple
partial derivative terms.
The final step involves updating model weights by using the computed 
$\frac{\partial E}{\partial W^{(l)}}$ with an update rule.
The exact update rule depends on the optimizer.

A notorious problem for training deep neural networks is the vanishing/exploding gradient
problem~\cite{bengio1993problem} that typically occurs in the backpropagation step when some of partial gradient terms in Eq.~\ref{eq.bprop} includes values larger or smaller than 1.
In this case, due to the multiple consecutive multiplications, the gradients \textit{w.r.t.} weights can get exponentially very small (close to 0) or very large (close to infinity) and
prevent effective learning of network weights.


%


Figures~\ref{fig:grad_flow_08} and \ref{fig:grad_flow_38} depict the gradient flows through VGG architectures \cite{simonyan2014very} with 8 and 38 layers respectively, trained and evaluated for a total of 100 epochs on the CIFAR100 dataset. \questionOne.


\section{Background Literature}
\label{sec:lit_rev}
In this section we will highlight some of the most influential
papers that have been central to overcoming the VGP in
deep CNNs.

\paragraph{Batch Normalization}\cite{ioffe2015batch}
BN seeks to solve the  problem of 
internal covariate shift (ICS), when distribution of each layer’s 
inputs changes during training, as the parameters of the previous layers change. 
The authors argue that without batch normalization, the distribution of
each layer’s inputs can vary significantly due to the  stochastic nature of randomly sampling mini-batches from your
training set. 
Layers in the network hence must continuously adapt to these high variance distributions which hinders the rate of convergence gradient-based optimizers.
This optimization problem is exacerbated further with network depth due
to the updating of parameters at layer $l$ being dependent on
the previous $l-1$ layers.

It is hence beneficial to embed the normalization of
training data into the network architecture after work from
LeCun \emph{et al.} showed that training converges faster with
this addition \cite{lecun2012efficient}. Through standardizing
the inputs to each layer, we take a step towards achieving
the fixed distributions of inputs that remove the ill effects
of ICS. Ioffe and Szegedy demonstrate the effectiveness of
their technique through training an ensemble of BN
networks which achieve an accuracy on the ImageNet classification
task exceeding that of humans in 14 times fewer
training steps than the state-of-the-art of the time.
It should be noted, however, that the exact reason for BN’s effectiveness is still not completely understood and it is 
an open research question~\cite{santurkar2018does}.



\paragraph{Residual networks (ResNet)}\cite{he2016deep} A well-known way of mitigating the VGP is proposed by He~\emph{et al.} in \cite{he2016deep}. In their paper, the authors depict the error curves of a 20 layer and a 56 layer network to motivate their method. Both training and testing error of the 56 layer network are significantly higher than of the shallower one.

\questionTwo.

Residual networks, colloquially
known as ResNets, aim to alleviate VGP through the
incorporation of skip connections that bypass the linear
transformations into the network architecture. 
The authors argue that this new mapping is significantly easier
to optimize since if an identity mapping were optimal, the
network could comfortably learn to push the residual to
zero rather than attempting to fit an identity mapping via
a stack of nonlinear layers. 
They bolster their argument
by successfully training ResNets with depths exceeding
1000 layers on the CIFAR10 dataset.
Prior to their work, training even a 100-layer was accepted
as a great challenge within the deep learning community.
The addition of skip connections solves the VGP through
enabling information to flow more freely throughout the
network architecture without the addition of neither extra
parameters, nor computational complexity.

\section{Solution overview}
\subsection{Batch normalization}





BN has been a standard component in the state-of-the-art 
convolutional neural networks \cite{he2016deep,huang2017densely}.
% As mentioned in Section~\ref{sec:lit_rev}, 
Concretely, BN is a
layer transformation that is performed to whiten the activations
originating from each layer. 
As computing full dataset statistics at each training iteration
would be computationally expensive, BN computes batch statistics
to approximate them. 
Given a minibatch of $B$ training samples and their feature maps
 $X = (\bx^1, \bx^2,\ldots , \bx^B)$ at an arbitrary layer where $X \in \mathbb{R}^{B\times H \times W \times C}$, $H, W$ are the height, width of the feature map and $C$ is the number of channels, the batch normalization first computes the following statistics:

\begin{align}
\label{eq.bnstats}
    \mu_c &= \frac{1}{BWH}  \sum_{n=1}^{B}\sum_{i,j=1}^{H,W} \bx_{cij}^{n}\\
    \sigma^2_c &= \frac{1}{BWH}
    \sum_{n=1}^{B}\sum_{i,j=1}^{H,W} (\bx_{cij}^{n} - \mu_{c})^2
\end{align} where $c$, $i$, $j$ denote the index values for $y$, $x$ and channel coordinates of feature maps, and $\bm{\mu}$ and $\bm{\sigma}^2$ are the mean and variance of the batch.

BN applies the following operation on each feature map in batch B for every $c,i,j$:
\begin{equation}
\label{eq.bnop}
\text{BN}(\bx_{cij}) = \frac{\bx_{cij} - \mu_{c}}{\sqrt{\sigma^2_{c}} + \epsilon} * \gamma_{c} + \beta_{c}
\end{equation} where $\gamma \in \mathbb{R}^C$ and $\beta\in \mathbb{R}^C$ are learnable parameters and $\epsilon$ is a small constant introduced to ensure numerical stability.

At inference time, using batch statistics is a poor choice as it introduces noise in the evaluation and might not even be well defined. Therefore, $\bm{\mu}$ and $\bm{\sigma}$ are replaced by running averages of the mean and variance computed during training, which is a better approximation of the full dataset statistics.

Recent work
has shown that BatchNorm has a more fundamental
benefit of smoothing the optimization landscape during
training \cite{santurkar2018does} thus enhancing the predictive
power of gradients as our guide to the global minimum.
Furthermore, a smoother optimization landscape should
additionally enable the use of a wider range of learning
rates and initialization schemes which is congruent with the
findings of Ioffe and Szegedy in the original BatchNorm
paper~\cite{ioffe2015batch}.


\subsection{Residual connections}

Residual connections are another approach used in the state-of-the-art Residual Networks~\cite{he2016deep} to tackle the vanishing gradient problem.
Introduced by He et. al.~\cite{he2016deep}, a residual block consists of a
convolution (or group of convolutions) layer, ``short-circuited'' with an identity mapping.
More precisely, given a mapping $F^{(b)}$ that denotes the transformation of the block $b$ (multiple consecutive layers), $F^{(b)}$ is applied to its input
feature map $\bx^{(b-1)}$ as $\bx^{(b)} = \bx^{(b-1)} + {F}(\bx^{(b-1)})$.

Intuitively, stacking residual blocks creates an architecture where inputs of each blocks
are given two paths : passing through the convolution or skipping to the next layer. A residual network can therefore be seen as an ensemble model averaging every sub-network
created by choosing one of the two paths. The skip connections allow gradients to flow
easily into early layers, since 
\begin{equation}
    \frac{\partial \bx^{(b)}}{\partial \bx^{(b-1)}} = \mathbbm{1} + \frac{\partial{F}(\bx^{(b-1)})}{\partial \bx^{(b-1)}}
    \label{eq.grad_skip}
\end{equation} where $\bx^{(b-1)} \in \mathbb{R}^{C \times H \times W }$ and $\mathbbm{1}$ is a $\mathbb{R}^{C \times H \times W}$-dimensional tensor with entries 1 where $C$, $H$ and $W$ denote the number of feature maps, its height and width respectively. 
Importantly, $\mathbbm{1}$ prevents the zero gradient flow.


\section{Experiment Setup}

\questionFigureFour

\questionFigureFive

\questionTableOne

We conduct our experiment on the CIFAR100 dataset \cite{krizhevsky2009learning},
which consists of 60,000 32x32 colour images from 100 different classes. The number of samples per class is balanced, and the
samples are split into training, validation, and test set while
maintaining balanced class proportions. In total, there are 47,500; 2,500; and 10,000 instances in the training, validation,
and test set, respectively. Moreover, we apply data augmentation strategies (cropping, horizontal flipping) to improve the generalization of the model.

With the goal of understanding whether BN or skip connections
help fighting vanishing gradients, we first test these
methods independently, before combining them in an attempt
to fully exploit the depth of the VGG38 model.

All experiments are conducted using the Adam optimizer with the default
learning rate (1e-3) -- unless otherwise specified, cosine annealing and a batch size of 100
for 100 epochs. 
Additionally, training images are augmented with random 
cropping and horizontal flipping.
Note that we do not use data augmentation at test time.
These hyperparameters along with the augmentation strategy are used
to produce the results shown in Fig.~\ref{fig:curves}.

When used, BN is applied
after each convolutional layer, before the Leaky
ReLU non-linearity. 
Similarly, the skip connections are applied from 
before the convolution layer to before the final activation function
of the block as per Fig.~2 of \cite{he2016deep}. 
Note that adding residual connections between the feature maps before and after downsampling requires special treatment, as there is a dimension mismatch between them. 
Therefore in the coursework, we do not use residual connections in the down-sampling blocks. However, please note that batch normalization should still be implemented for these blocks. 

\subsection{Residual Connections to Downsampling Layers}
\label{subsec:rescimp}

\questionThree.


\section{Results and Discussion}
\label{sec:disc}

\questionFour.

\section{Conclusion}
\label{sec:concl}

\questionFive.    

\bibliography{refs}

\end{document} 





